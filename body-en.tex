
\section{Introduction}
\label{sec:introduction}

\nota{Context}

In the past few years we have been developing Rameau, a system for
automatic harmonic analysis \cite{kroger08:rameau}. Rameau, as of the
last update, had support for many different chord-labeling algorithms,
automatic identitication of non-chordal sonorities, basic
musicological functionality (such as cadence and voice crossing
detection), Lilypond \cite{nienhuys.ea08:lilypond} integration, tonal
codification support and many other interesting properties.

\nota{Focus}

Recently, we have implemented in Rameau support for roman numeral
functional harmonic analysis. This includes adding notions of
keyfinding, tonal function detection, modulation and tonicization
detection and some other secondary features. Supporting this analysis
involves many important musicological decisions, and has some
non-trivial consequences. For example, many tonal pieces have phrases
and even movements that are non-harmonical in nature, such as some
contrapuntual passages in Brahms' no 1 quartet \nota{confirmar, citar
  direito}. Every harmonic analysis of such a passage is nonsensical,
and, if automatically performed, will probably find cadences in the
noise instead of detecting the structure of the piece. Another
important problem is how to represent, internally, a roman numeral
functional analysis. How we approached these problems will be made
clear in the remainder of this paper.

\nota{Related work}

\cite{raphael.ea03:harmonic} \cite{maxwell92:expert}
\cite{tsui02:harmonic} \cite{temperley.ea99:modeling}
\cite{taube99:automatic} \cite{winograd68:linguistics}
\cite{barthelemy.ea01:figured} \cite{ulrich77:analysis}

\nota{contributions}

In this paper we present the infrastructure and algorithms implemented
in Rameau to properly do functional harmonic analysis, and the
problems encountered on the path.

\nota{structure}

\section{The problem}
\label{sec:problem}

Roman numeral functional analysis consists, roughly, in two
activities: key finding and roman numeral function detection. This
view, on the one hand, highlight the important factors for this
specific problem but, on the other hand, ignores important sub-aspects
of harmonic analysis such as the detection of harmonically significant
segments, the disambiguation of enharmonic notes, the disambiguation
of possibly different chord types (such as telling a major chord
without a third apart from a minor chord without a third).

\subsection{Key finding}
\label{sec:key-finding}

\nota{keyfinding é algo terrivelmente ambíguo, pode ter vários níveis
  de verdade}

The concept of key, in tonal music theory, is vague and ambiguous, and
telling its many meanings apart is crucial when designing harmonic
analysis software. When speaking of key, a musician might speak of the
key of a piece, the key of a passage (a movement in a symphony, for
example, might have a different key from the main key in the
symphony), the key of a phrase, the key of a cadence, and even the key
of a single chord (for example, it might be argued that, for a
secondary dominant chord, its key might be said to be different from
the overall key of the surrounding passage). Clearly, there might be
more or less different scopes for the key of a song, and determining
in which scope (or scopes) to perform analysis. There's also the
problem of determining how correct the key an algorithm finds for a
given chord is.

\nota{como mostrar o resultado}

Another important problem involving keyfinding is the graphical
representation of a key that is shown to the user.

\subsection{Roman function detection}
\label{sec:roman-funct-detect}

\nota{funções romanas dependem muito da tonalidade encontrada}

\nota{e dependem também das notas de forma sutil}

\section{The Framework}
\label{sec:framework}

\nota{gabarito}

The first, and arguably most important, implementation decision when
desigining the roman numeral functional analysis part of Rameau was
the internal format for representing a complete analysis of a
piece. As described in \cite{kroger08:rameau}, Rameau represents a
piece internally as a list of notes and, before analysis, converts it
into a list of sonorities (where a sonority is a set of notes sounding
at a given instant of time). Therefore, we chose the format for the
roman numeral functional analysis to be a list of instantaneous
analyzes. An instantaneous analysis is a snapshot of the analysis for
one sonority. Due to the key granularity problem described in section
\ref{sec:key-finding}, this snapshot only shows the most local
information possible. Later on we will extend this format to encompass
notions such as the overall key of a piece or a movement. Each such
snapshot is a tuple $(K,R)$, where $K$ is the key and $R$ is the roman
numeral analysis for that sonority under key $K$. Internally, a key is
a tuple $(P_k,M_k)$, where $P_k$ is the key's pitch and $M_k$ is the
key's mode (which can be major or minor). The roman function $R$ is
also a tuple $(N,A,M_r)$, where $N$ is the roman numeral (when
applicabble), $A$ is the number of sharps (of flats) on that numeral
and $M_r$ is the roman function's mode. For example, a minor flat
seventh chord in a $D$ major key is represented internally as
$((D,major),(vii,\flat,minor)$. While this representation has some
drawbacks, such as the impossibility of clearly marking non-harmonic
passages, it is extensible and practical for analysing simple
harmonies such as the Bach chorales.

\nota{conferir com resultados}

One of the goal of the Rameau project is understanding and comparing
previously published and new techniques for automatic harmonic
analysis. To properly compre harmonic analysis algorithms we have a
corpus of 371 Bach chorales from the Riemenschneider edition
\cite{bach41:371}, 20 of them annotated by experts with an acceptable
harmonic analysis. Rameau includes facilities to train machine
learning algorithms on these analyzed chorales and a bayesian
framework for estimating the correctness of the expert annotations and
using this estimate to derive confidence intervals for the accuracies
of the algorithms \nota{citar monografia?}. The analysis results can
also be neatly typeset with the aid of the Lilypond music typesetting
program \cite{nienhuys.ea08:lilypond}.

\section{Algorithms}
\label{sec:algorithms}

Using the infrastructure described in section \ref{sec:framework}
Rameau currently has implementations of four different roman numeral
functional analysis; a hidden Markov model, a k-nearest neighbors
classifier and a trivial extension of Pardo \& Birmingham's chord
labeling algorithm \cite{pardo.ea99:automated}.

\subsection{Hidden Markov Model}
\label{sec:hidden-markov-model}

A hidden Markov model is any probabilistic function of a Markov
chain. Using a hidden Markov model to perform roman numeral function
analysis consists of modeling the notes in a tonal piece as a
probabilistic function of the underlying harmonies, and finding these
harmonies, given the notes, using traditional hidden Markov model
algorithms. Our approach closely follows that of Raphael and Stoddard
\cite{raphael.ea03:harmonic}, and the differences will be noted here.

The underlying Markov chain is assumed to be from a pair $(K,R)$ (as
described in section \ref{sec:framework}) to another such
pair. Internally, though, this transition table is compressed assuming
some invariances (without this compression we would need around $10^9$
entries in it). The only invariance we assume is that transposing the
music to another key with the same mode should not change its
analysis. The notes found in the actual piece are assumed to be
independently and identically drawn from a distribution particular to
each possible key and roman function. This table is also compressed
assuming a transposition invariance on the key pitch. 

A good way to see how much of the underlying structure of a process is
captured by a hidden Markov model is by analysing samples from the
model. Figure \ref{fig:amostra} shows such a sample. As it can be
seen, the Markov model has no concept of voice leading and is quite
liberal in using non-chord tones. The line marked as ``Original''
shows the undelying state sequence used by the hidden Markov model to
generate this piece. The line labeled ``HMM Analysis'' is an analysis
of the same piece by the HMM. It already corrects some misconceptions
of the generating model about chord structure. Finally, the line
labeled ``Correct Analysis'' shows the best a human expert could do to
understand the underlying harmony of this piece. This sample also
shows an important flaw of this technique, which is the high prior
probability given to a piece that ends in an entirely different key
from the one in which it started. Perhaps some notion of a ``global''
or ``underlying'' key might improve the performance of this algorithm.

\begin{figure*}[t]
  \centering
  \includegraphics[trim=0cm 5cm 0cm 2.5cm]{amostra-markov}
  \caption{A sample from the hidden Markov model}
  \label{fig:amostra}
\end{figure*}

\subsection{KNN}
\label{sec:knn}



\subsection{Pardo \& Birmingham's}
\label{sec:pardo--birminghams}

\section{Example analyses}
\label{sec:example-analyses}

\section{Conclusions and future work}
\label{sec:concl-future-work}


%%% Local Variables: 
%%% mode: latex
%%% TeX-master: "icmc2009"
%%% End: 
