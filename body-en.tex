
\section{Introduction}
\label{sec:introduction}

\nota{Context}

In the past few years we have been developing Rameau, a system for
automatic harmonic analysis \cite{kroger08:rameau}. Rameau, as of the
last update, had support for many different chord-labeling algorithms,
automatic identitication of non-chordal sonorities, basic
musicological functionality (such as cadence and voice crossing
detection), Lilypond \cite{nienhuys.ea08:lilypond} integration, tonal
codification support and many other interesting properties.

\nota{Focus}

Recently, we have implemented in Rameau support for roman numeral
functional harmonic analysis. This includes adding notions of
keyfinding, tonal function detection, modulation and tonicization
detection and some other secondary features. Supporting this analysis
involves many important musicological decisions, and has some
non-trivial consequences. For example, many tonal pieces have phrases
and even movements that are non-harmonical in nature, such as some
contrapuntual passages in Brahms' no 1 quartet \nota{confirmar, citar
  direito}. Every harmonic analysis of such a passage is nonsensical,
and, if automatically performed, will probably find cadences in the
noise instead of detecting the structure of the piece. Another
important problem is how to represent, internally, a roman numeral
functional analysis. How we approached these problems will be made
clear in the remainder of this paper.

\nota{Related work}

\cite{raphael.ea03:harmonic} \cite{maxwell92:expert}
\cite{tsui02:harmonic} \cite{temperley.ea99:modeling}
\cite{taube99:automatic} \cite{winograd68:linguistics}
\cite{barthelemy.ea01:figured} \cite{ulrich77:analysis}

\nota{contributions}

In this paper we present the infrastructure and algorithms implemented
in Rameau to properly do functional harmonic analysis, and the
problems encountered on the path.

\nota{structure}

\section{The problem}
\label{sec:problem}



\section{The Framework}
\label{sec:framework}


\section{Algorithms}
\label{sec:algorithms}

\subsection{Hidden Markov Model}
\label{sec:hidden-markov-model}

\subsection{KNN}
\label{sec:knn}

\subsection{Pardo \& Birmingham's}
\label{sec:pardo--birminghams}

\section{Example analyses}
\label{sec:example-analyses}

\section{Conclusions and future work}
\label{sec:concl-future-work}


%%% Local Variables: 
%%% mode: latex
%%% TeX-master: "icmc2009"
%%% End: 
