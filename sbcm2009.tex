\documentclass[12pt]{article}
\usepackage{sbc-template,amssymb,amsmath} 
\usepackage{times}
\usepackage{amsmath}
\usepackage{microtype}
\usepackage{amsthm}
\usepackage{url}
\usepackage[utf8x]{inputenc}
\usepackage[T1]{fontenc}
\usepackage{color}
\usepackage{subfig}
\usepackage{mathptmx} 
\usepackage[ruled]{algorithm2e}
\usepackage[pdftex,colorlinks=false,bookmarksnumbered,pdfstartview=XYZ]{hyperref}
\usepackage{graphicx} % for compatible graphics with hyperref
\usepackage[figure,table]{hypcap} % corrects the hyper-anchor of figures/tables


\newcounter{notecounter}
\newcommand{\nota}[1]{\addtocounter{notecounter}{1}{\textcolor{red}{[nota
      \arabic{notecounter}: #1]}}}

\title{Functional Harmonic Analysis and Computational Musicology in
  Rameau}

\author{Alexandre Tachard Passos\inst{1}, Marcos Sampaio\inst{2},
  Pedro Kröger\inst{2}, Givaldo de Cidra\inst{2}}

\address{Instituto de Computação (IC) \\
Universidade Estadual de Campinas (UNICAMP) \\
Campinas, Brazil
\nextinstitute
Genos---Computer Music Research Group \\
School of Music (EMUS)\\
Federal University of Bahia (UFBA) \\
Salvador, Brazil}

\begin{document}


\maketitle

\begin{abstract}
In the paper we present the infrastructure for computational
musicology and functional harmonic analysis in Rameau, a framework for
experimentation with musicological ideas in software. Rameau supports
out of the box chord labeling, key finding, tonal function detection,
cadence detection, voice crossing identification, parallel fifths and
octaves recognition, seventh note resolution analysis, and can be
easily extended to support many other features. It can also generate
textual reports, graphical visualization, and typeset scores with the
results of these analyses. Rameau is fully open source and implemented
in Common Lisp.

%%% Local Variables: 
%%% mode: latex
%%% TeX-master: "icmc2009"
%%% End: 

\end{abstract}

\section{Introduction}

This template includes all the information about formatting manuscripts
for the ICMC 2009. Please follow these guidelines to give the final
proceedings a uniform look. 

\cite{allan02:harmonising}

%%% Local Variables: 
%%% mode: latex
%%% TeX-master: "icmc2009"
%%% End: 


\bibliographystyle{sbc}
\bibliography{strings-short,genos,ismir,programs,coding,harmonic-analysis,dont-have,artificial-inteligence,music-harmony-and-theory,licenses,icmc,music-scores,computational-musicology}

\end{document}
